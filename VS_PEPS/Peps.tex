\documentclass[french,12pt,a4paper]{article}
\usepackage[T1]{fontenc}
\usepackage[utf8]{inputenc}
\usepackage[dvips]{graphicx}
%\usepackage[english]{babel}
\usepackage[frenchb]{babel}
\AddThinSpaceBeforeFootnotes % à insérer si on utilise \usepackage[french]{babel}
\FrenchFootnotes % à insérer si on utilise \usepackage[french]{babel}
\usepackage{amsmath,amsthm,amsfonts,amssymb}
\usepackage{mathrsfs}
\usepackage{array}
\usepackage{color}
\usepackage{float}
\usepackage{pstricks,pstricks-add,pst-plot,pst-tree}
\usepackage{enumerate}
\usepackage{textcomp}
\usepackage{lettrine}
\usepackage{lscape}
\usepackage{lmodern}
\usepackage{stmaryrd}
\usepackage{subfigure}
\usepackage{multido}
\usepackage{listings}
\usepackage{footnote}
\usepackage{appendix}
\usepackage{color}
\usepackage{listings}
\usepackage{tikz}
\usepackage{dsfont}
\usetikzlibrary{matrix}

\lstset{
  language={Matlab},
  numbers=left, numberstyle=\tiny, stepnumber=1, firstnumber=last,
  frameround=tttt, 
  frame=single, 
  float,
  captionpos=b,
  breaklines=true,
  sensitive=f,
  morestring=[d]",
  basicstyle=\small\ttfamily,
  keywordstyle=\bf\small,
  stringstyle=\sf
}
\usepackage{fancyhdr,lastpage}
\usepackage[twoside,left=2cm,top=2.5cm,dvips,marginparwidth=1.9cm,marginparsep=0.5cm,headheight=35pt]{geometry}
%En tete et pied de page
\usepackage{fancyhdr}
\pagestyle{fancy}
\lhead{\leftmark} 
\chead{}
\rhead{PEPS}
\lfoot{ENSIMAG 3A}
\cfoot{\textit{Equipe 7}}
\rfoot{\thepage}
\renewcommand{\headrulewidth}{0pt}  
\renewcommand{\footrulewidth}{0.4pt}
\title{Projet .NET : Gestion indicielle}
\date{October 31, 475}
\author{Guillaume Fuchs, Guillaume Pelletier, Samuel Rosilio}
%Page de garde

\begin{document}
\begin{titlepage}
\begin{center}

\textsc{\LARGE ENSIMAG}\\[1.5cm]

\textsc{\Large Projet d'évaluation de produit structuré}\\[0.5cm]

% Title
 \hrule
 \hrule 
 
\vspace{7mm}
{ \huge \bfseries Playlist 2  }

\vspace{7mm}
\hrule
\hrule

\vspace{7mm}
% Author and supervisor
\begin{minipage}{0.4\textwidth}
\begin{flushleft} \large
\emph{Etudiants:}\\
Guillaume \textsc{Fuchs},\\
Guillaume \textsc{Pelletier},\\
Samuel \textsc{Rosilio}
\end{flushleft}
\end{minipage}

\vfill

% Bottom of the page
{\large \today}

\end{center}
\end{titlepage}
\tableofcontents
\newpage

\section{Introduction}



\section{Définition}

Le \lstinline!depositaire! du produit financier est l'organisme qui sera chargé de :\\
\begin{itemize}
\item[•]
	La garde des avoirs en dépôt et leur restitution.
\item[•]
	Le dépouillement des ordres.
\item[•]
	L'information de la société de gestion ou de la SICAV des opérations relatives aux titres conservés pour son compte.
\item[•]
	Le contrôle de la régularité des décisions de l'OPC ou de se société de gestion par rapport aux dispositions législatives et réglementaires applicables.\\
\end{itemize}


La \lstinline!societe de gestion! associé au produit devra s'occuper de :\\
\begin{itemize}
\item[•]
	La gestion de portefeuille pour le compte de tiers consiste à gérer des portefeuilles individuels d'instruments financiers pour le compte de clients, qu'il s'agisse par exemple de clients particuliers ou d'investisseurs institutionnels. Un mandat de gestion est conclu entre la société de gestion et son client.
\item[•]
	La gestion collective ou gestion d’organismes de placement collectif (OPC) consiste schématiquement à gérer des portefeuilles collectifs. Un OPC est constitué des sommes mises en commun par des investisseurs et gérées pour leur compte par un gestionnaire de portefeuille. Ce dernier utilise ces sommes pour acquérir des instruments financiers, par exemple des actions ou des obligations en fonction de ses objectifs. Des parts ou des actions représentant une quote-part de l’avoir de l’OPC sont émises, en contrepartie des sommes versées dans l’OPC. \\
\end{itemize}


Un FCP fait parti de la famille des OPCVM.\\
\begin{itemize}


\item[•]
OPCVM: entité qui gère un portefeuille dont les fonds sont placés en valeurs mobilières. 

Garantie de l’investissement.

\item[•]
Indice boursier: Mesure statistique calculée par le regroupement des valeurs de titres de 

plusieurs sociétés.

Attention les indices des grands marchés mondiaux sont de plus en plus corrélés entre eux.

\item[•]
OAT: obligation assimilable du Trésor, emprunts d’Etat émis pour une durée de 5 à 50 ans.

\item[•]
Valeur liquidative: division de l’actif net de l’OPCVM par son nombre de parts

Calculé toutes les 2 semaines => actif<80millions €

\item[•]
Performance: Plus ou moins-value réalisée par rapport à l’investissement initial.

Performance avec dividendes réinvestis:les dividendes sont réinvestis le jour même pour 

souscrire des actions supplémentaire. Permet d’estimer l’évolution véritable de la valeur de 

l’OPCVM, indépendamment de son mode de distribution. 

\item[•]
Fonds à formule: regroupe plusieurs catégories d’OPCVM. Offre une perspective de 

gain dépendant des évolutions des marchés financiers, selon des paramètres définis à la 

souscription. Offre une garantie sur le capital initialement investi

Gestion active du fonds dans des actifs non-risqués (monétaires, obligataires …) pour garantir 

le capital et des actifs risqués (actions, dérivés…) à la recherche de performance.

\item[•]
OCDE: organisation de coopération et de développement économiques. Publication d’études, 

statistiques. Membres ont un système de gouvernement démocratique et une économie de 

marché.

\item[•]
Caisse d’amortissement de la dette sociale: organisme gouvernemental français qui possède 

la dette sociale.

\item[•]
BPCE: emsemble des entreprises qui composent la Caisse nationale des Caisses d’épargne et 

de la Banque fédéral des Banques populaires.

\item[•]
Cession temporaire des titres: vente de titres contre espèces ou autres titres avec un 

engagement irrévocable de part et d’autres de restituer les valeurs échangées.
\end{itemize}

\section{Analyse des flux financiers du produit}

\subsection{Description du produit}

Notre produit financier prend appui sur l'évolution de 4 indices boursiers à travers le monde, les flux versés dépendront en effet des rendements de chaque indice depuis le 29 avril 2010.
Les quatre indices considérés sont le Footsie 100, le Standard \& Poor's 500, Dow Jones Euro Stoxx 50 et le Nikkei 225.\\
Le fonctionnement de notre produit est alors le suivant, si 25 avril 2011 au moins trois des quatres indices ont eu un rendement supérieur à 10\% alors un flux de 4,5\% de la valeur nominale nous sera versé à échéance. De plus si trois des quatre indices ont eu un rendement supérieur à 20\% alors la date d'échéance devient le 25 avril 2011 et nous recevons 104,5\% de la valeur nominale.\\
En date du 23 avril 2012, nous raisonnons de la même façon en effet si trois des quatre indices ont eu un rendement supérieur à 10\% depuis le 29 avril 2010 alors on obtiendra 4,5\% de plus que ce que l'on devra déjà acquérir. En revanche si trois des quatres indices ont dépassé 20\% de rendement, la date d'échéance devient le 23 avril 2012.\\
Pour ce qui est des années 2013, 2014, 2015 et 2016, on accumule simplement 4,5\% en plus à échéance si trois des quatre indices ont un rendement depuis 2010 supérieur à 10\%. De plus la date d'échéance est le 25 avril 2016.\\

\subsection{Flux envisageables du produit}

Commençons par envisager les différents cas possible :\\
Soit $$ A = \left\{ Dow Jones Euro Stoxx 50, Standard \& Poor's 500, Footsie 100, Nikkei 225 \right\}$$\\
Ainsi $$\forall i, j, k, l \in A tels que A = \left\lbrace i, j, k, l \right\rbrace$$\\

\begin{itemize}

\item[•]
Cas où l'échéance est en fin de première année :\\

\begin{center}
\begin{tabular}{|c|c|c|c|c|c|}
  \hline
  Date & R_{i} & R_{j} & R_{k} & R_{l} & Prime accumulée \\
  \hline
  25/04/2011 & 21\% & 15.6\% & 22.6\% & 31.1\% & 4.5\%\\
  \hline
\end{tabular}
\end{center}

On constate que dès la première année que trois des quatre indices dépassent un rendement de 20\% ainsi le produit arrive à échéance en fin de première année et recevra 104,5\% puisque par conséquent trois des quatre indices ont eu une performance supérieur à 10\% \\

\item[•]
Cas le moins intéressant pour le client avec échéance deux ans :\\

\begin{center}
\begin{tabular}{|c|c|c|c|c|c|}
  \hline
  Date & R_{i} & R_{j} & R_{k} & R_{l} & Prime accumulée \\
  \hline
  25/04/2011 & 5.8\% & 15.6\% & 2.6\% & -7.1\% & 0\%\\
  23/04/2012 & 21.3\% & 28.8\% & 20.4\% & 10.3\% & 4.5\%\\
  \hline
\end{tabular}
\end{center}

Dans ce cas ci, aucune prime n'est acquise grâce aux performances des indices la première année, de plus la seconde année comme trois des quatre performance dépasse 20\%, l'échéance devient le 23/04/2012 et 104,5\% sera versé au client.

\item[•]
Cas le plus intéressant pour le client avec échéance deux ans :\\

\begin{center}
\begin{tabular}{|c|c|c|c|c|c|}
  \hline
  Date & R_{i} & R_{j} & R_{k} & R_{l} & Prime accumulée \\
  \hline
  25/04/2011 & 13.8\% & 15.6\% & 10.6\% & -7.1\% & 4.5\% \\
  23/04/2012 & 21.3\% & 28.8\% & 20.4\% & 5.3\% & 4.5\% \\
  \hline
\end{tabular}
\end{center}

Ainsi dans ce cas le client recevra une prime de 9\% de la valeur liquidative de référence en fin de deuxième année. L'échéance devient donc le 23/04/2012 et 109\% de la valeur liquidative de référence sera versé au client.\\

\item[•]
Nous allons maintenant distinguer les deux cas extrême dans le cas d'une échéance à six ans, le premier cas correspond à celui où le client ne gagnera rien soit :
\begin{center}
\begin{tabular}{|c|c|c|c|c|c|}
  \hline
  Date & R_{i} & R_{j} & R_{k} & R_{l} & Prime accumulée \\
  \hline
  25/04/2011 & 3.8\% & 5.6\% & 0.6\% & -7.1\% & 0\% \\
  23/04/2012 & 10.3\% & 8.8\% & -10.4\% & 5.3\% & 0\% \\
  29/04/2013 & 4.5\% & 12.4\% & 4.2\% & 14.2\% & 0\%\\
  29/04/2014 & -5.4\% & 16.3\% & 10.2\% & 9.4\% & 0\%\\
  29/04/2015 & 3.3\% & 11.1\% & 7.1\% & 18.2\% & 0\%\\
  25/04/2016 & 11.5\% & 15.2\% & 8.7\% & 9.9\% & 0\%\\
  \hline
\end{tabular}
\end{center}

Dans ce cas ci qui est le cas le plus défavorable sur un horizon de six ans, le client retouchera la valeur liquidative de référence à la fin de la sixième année.

\item[•]
Cas le plus favorable pour une échéance de six ans :
\begin{center}
\begin{tabular}{|c|c|c|c|c|c|}
  \hline
  Date & R_{i} & R_{j} & R_{k} & R_{l} & Prime accumulée \\
  \hline
  25/04/2011 & 13.8\% & 15.6\% & 10.6\% & -7.1\% & 4.5\% \\
  23/04/2012 & 10.3\% & 18.8\% & -10.4\% & 15.3\% & 4.5\% \\
  29/04/2013 & 14.5\% & 12.4\% & 4.2\% & 14.2\% & 4.5\%\\
  29/04/2014 & -5.4\% & 16.3\% & 10.2\% & 19.4\% & 4.5\%\\
  29/04/2015 & 3.3\% & 11.1\% & 17.1\% & 18.2\% & 4.5\%\\
  25/04/2016 & 21.5\% & 15.2\% & 18.7\% & 29.9\% & 4.5\%\\
  \hline
\end{tabular}
\end{center}

Dans ce cas ci, le client touchera à la fin de la sixième année 127\% de la valeur liquidative de référence.
\end{itemize}

On peut ainsi résumé avec l'arbre suivant les flux possibles à échéance du produit (en notant 0 les noeuds non terminaux)  :

% Set the overall layout of the tree
\tikzstyle{level 1}=[level distance=4cm, sibling distance=3cm,->]
\tikzstyle{level 2}=[level distance=4cm, sibling distance=2cm,->]

% Define styles for bags and leafs
\tikzstyle{bag} = [text width=2em, text centered]
\tikzstyle{end} = []

% The sloped option gives rotated edge labels. Personally
% I find sloped labels a bit difficult to read. Remove the sloped options
% to get horizontal labels. 
\begin{tikzpicture}[>=stealth,sloped]
    \matrix (tree) [%
      matrix of nodes,
      minimum size=1cm,
      column sep=3cm,
      row sep=0.9cm,
    ]
    {
          &       &      & 190.5\\
          &       & 163.5  & 183.75\\
          & 156.75 &      & 177 \\
      150 &       & 156.75& 170.25 \\
          & 0   &      & 163.5 \\
          &       & 0  & 156.75\\
          &       &      & 150 \\
    };
    \draw[->] (tree-4-1) -- (tree-3-2) node [midway,above] {};
    \draw[->] (tree-4-1) -- (tree-5-2) node [midway,below] {};
    \draw[->] (tree-3-2) -- (tree-2-3) node [midway,above] {};
    \draw[->] (tree-3-2) -- (tree-4-3) node [midway,below] {};
    \draw[->] (tree-5-2) -- (tree-4-3) node [midway,above] {};
    \draw[->] (tree-5-2) -- (tree-6-3) node [midway,below] {};
    \draw[->] (tree-2-3) -- (tree-1-4) node [midway,below] {};
    \draw[->] (tree-2-3) -- (tree-2-4) node [midway,below] {};
    \draw[->] (tree-2-3) -- (tree-3-4) node [midway,below] {};
    \draw[->] (tree-2-3) -- (tree-4-4) node [midway,below] {};
    \draw[->] (tree-2-3) -- (tree-5-4) node [midway,below] {};
    \draw[->] (tree-4-3) -- (tree-2-4) node [midway,below] {};
    \draw[->] (tree-4-3) -- (tree-3-4) node [midway,below] {};
    \draw[->] (tree-4-3) -- (tree-4-4) node [midway,below] {};
    \draw[->] (tree-4-3) -- (tree-5-4) node [midway,below] {};
    \draw[->] (tree-4-3) -- (tree-6-4) node [midway,below] {};
    \draw[->] (tree-6-3) -- (tree-3-4) node [midway,below] {};
    \draw[->] (tree-6-3) -- (tree-4-4) node [midway,below] {};
    \draw[->] (tree-6-3) -- (tree-5-4) node [midway,below] {};
    \draw[->] (tree-6-3) -- (tree-6-4) node [midway,below] {};
    \draw[->] (tree-6-3) -- (tree-7-4) node [midway,below] {};
  \end{tikzpicture}
  
  Est il nécessaire de le terminer ?
  
  \subsection{Rentabilités espérées du produit}
  
  On peut avoir de même un tableau résumant les rentabilités actuarielles annuelles sous la forme suivante en considérant que le prix de départ est de la valeur liquidative de référence plus la souscription :
  
   \begin{tikzpicture}[>=stealth,sloped]
    \matrix (tree) [%
      matrix of nodes,
      minimum size=1cm,
      column sep=3cm,
      row sep=1cm,
    ]
    {
          &       &      & 3.64\%\\
          &       & 3.12\%  & 3.02\%\\
          & 1.95\% &      & 2.37\% \\
        0 &       & 0.97\% & 1.71\% \\
          & 0   &      & 1.03\% \\
          &       & 0  & 0.32\%\\
          &       &      & -0.41\% \\
    };
    \draw[->] (tree-4-1) -- (tree-3-2) node [midway,above] {};
    \draw[->] (tree-4-1) -- (tree-5-2) node [midway,below] {};
    \draw[->] (tree-3-2) -- (tree-2-3) node [midway,above] {};
    \draw[->] (tree-3-2) -- (tree-4-3) node [midway,below] {};
    \draw[->] (tree-5-2) -- (tree-4-3) node [midway,above] {};
    \draw[->] (tree-5-2) -- (tree-6-3) node [midway,below] {};
    \draw[->] (tree-2-3) -- (tree-1-4) node [midway,below] {};
    \draw[->] (tree-2-3) -- (tree-2-4) node [midway,below] {};
    \draw[->] (tree-2-3) -- (tree-3-4) node [midway,below] {};
    \draw[->] (tree-2-3) -- (tree-4-4) node [midway,below] {};
    \draw[->] (tree-2-3) -- (tree-5-4) node [midway,below] {};
    \draw[->] (tree-4-3) -- (tree-2-4) node [midway,below] {};
    \draw[->] (tree-4-3) -- (tree-3-4) node [midway,below] {};
    \draw[->] (tree-4-3) -- (tree-4-4) node [midway,below] {};
    \draw[->] (tree-4-3) -- (tree-5-4) node [midway,below] {};
    \draw[->] (tree-4-3) -- (tree-6-4) node [midway,below] {};
    \draw[->] (tree-6-3) -- (tree-3-4) node [midway,below] {};
    \draw[->] (tree-6-3) -- (tree-4-4) node [midway,below] {};
    \draw[->] (tree-6-3) -- (tree-5-4) node [midway,below] {};
    \draw[->] (tree-6-3) -- (tree-6-4) node [midway,below] {};
    \draw[->] (tree-6-3) -- (tree-7-4) node [midway,below] {};
  \end{tikzpicture}
  
  \subsection{Expression analytique des flux}
  Soit $A = \left\lbrace$ Dow Jones Euro Stoxx $50$, Standard \& Poor's $500$, Footsie $100$, Nikkei $225$ $\right\rbrace$ \\
  
  On obtient alors :
  
  $$ F_{versé} = N_{0} \sum_{i=1}^{6}4.5\%*\mathds{1}_{\left\lbrace \sum_{I \in A} \mathds{1}_{\left\lbrace Perf_{I}(i)>10\% \right\rbrace}>2 \right\rbrace} * \sum_{j=1}^{\min{1,2}} \mathds{1}_{\left\lbrace \sum_{I \in A} \mathds{1}_{\left\lbrace Perf_{I}(i)>20\% \right\rbrace}<3 \right\rbrace} $$
  
\section{Analyse des risques du produit}

\end{document}
